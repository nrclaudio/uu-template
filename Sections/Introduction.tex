\documentclass[../main.tex]{subfiles}

 
\begin{document}
\subsection{Inflammatory Bowel Diseases}

The term Inflammatory Bowel Disease (IBD) refers to a set of chronic inflammatory diseases of the gastroenteric tissue. These are comprised of Ulcerative Colitis (UC) and Chron's disease (CD). Both diseases are defined by cycling phases of relapse and remission of intestinal inflammation. While CD can affect any part of the tract, UC typically involves the colon \citep{wallace_immunopathology_2014}. 

\subsubsection{Generalities}
As many other immune diseases, IBDs are characterized by an immune deregulation. In this case, that results in an abnormal response against commensal microbiota in patients with a predisposed genetic background. Recent studies trying to unravel the pathogenesis of the disease revealed its association with environmental factors, genetics and the microbiome. 

Among the environmental factors involved, antibiotic use \citep{hviid_antibiotic_2011} and diet \citep{de_filippo_impact_2010} have been reported to be associated with the disease. Furthermore, the surge of Genome Wide Association Studies (GWAS) in the late 2000s unveiled genetic variants that are associated with an increase in the susceptibility to IBDs. This variants play a role in innate and adaptive immunity regulatory networks \citep{duerr_genome-wide_2006,hampe_genome-wide_2007,burton_association_2007}.  

When being born, a human's gastrointestinal tract is colonized by a broad range of microorganisms. These microorganisms eclipse host cells by approximately 10 fold \citep{saleh_experimental_2011}, creating a vast, complex network where immune cells and the microbiota interact with each other. The tolerance as well as the regulation of these interactions is key for the intestinal homeostasis \citep{nell_impact_2010}. It's been widely described how the host micriobiota serves as stimulus for an inflammatory response leading to IBD \citep{dhaens_early_1998}.  

It's still unclear whether the cause of IBD is an imbalance in bacterial content or a deregulated immune response against the microbiota, but it probably is an interplay of these and other unknown factors.  

\subsubsection{Relevance}

 During the second half of the 20th century, an increase in the incidence of UC and CD took place in western countries \citep{molodecky_increasing_2012}. Recent studies have shown that the same shift is happening in countries with rapid socioeconomic growth \citep{kaplan_globalisation_2016}.
 
 Nowadays, more than 2.5 million people suffer from the disease worldwide \citep{burisch_burden_2013}. In Europe, the incidence of IBDs was 505 per 100.000 persons for UC and 322 per 100.000 persons for CD. In North America, the extent of UC was reported to be of 286.3 cases per 100.000 persons and 318.5 cases per 100.000 persons in the case of CD \citep{ng_worldwide_2017}.

\subsubsection{Current treatment}
Since the pathology of the disease hasn't been characterised yet, today's treatment tries to maintain the patient in
remission and mitigate secondary effects, rather than correcting or reversing the mechanism. 

%Among the drugs used we can find Aminosalicylates, Corticosteroids, Immunosuppressants, Cyclosporines and Monoclonal antibodies \citep{hanauer_medical_1994}.

\subsection{The immune system and IBD}

As it has been stated above, IBDs are characterised by an immune deregulation. This phenomenon is comprised of an epithelial damage \citep{korzenik_evolving_2006}, usually related to an anomalous mucus production or defective repair; the development of inflammation induced by the micriobiota and immune cells infiltrating the lamina propia (LP) \citep{choy_overview_2017}; and the failure of the immune regulation that controls the aforementioned response \citep{ince_immunologic_2007}. 

The gastrointestinal immune system consists of innate and adaptive immunity. The innate immune system is comprised of the intestinal mucin, the epithelium, an acidic pH, Macrophages, Neutrophils, Dendritic cells (DCs), Innate Lymphoid Cells (ILCs) as well as cytokines and other immune molecules such interleukins (ILs) and tumor necrosis factors (TNFs). When this innate response is not able to prevent the pathogenesis, the adaptive immune system starts producing pathogen-specific B and T cells that will help overcome the disease.

It's been shown that mice lacking Recombination-activating genes (RAG\textsuperscript{-/-}), hence lacking an adaptive immune system, don't develop spontaneous colitis. Notwithstanding, when these mice are treated with chemicals or pathogens that leverage colitis-like conditions, they develop colitis \citep{buonocore_innate_2010}. This suggests that the components of the innate immune system, in the abscence of an adaptive immune system, can develop IBD. Nonetheless, the adaptive immune system is thought to play a role in the immunopathology of the disease.

All in all, the pathology of IBD is tightly regulated. Whether an immune tolerance or a defensive inflammatory response takes place highly depends on the appropiate functioning of this regulation. 

\subsubsection{Innate immune system}

The innate immune system is the first defensive measure against foreign pathogens and anomalies and triggers the adaptive immune response. This response is non-specific and has no memory.

One of the first effectors of the innate immune response is the epithelium. There are four types of Intestinal Epithelial Cells (IECs): enterocytes, forming the focal surface of the small and large intestine; goblet cells, which produce the mucus; enteroendocrine cells that secrete hormones; and Paneth cells, which synthetize antimicrobial peptides and proteins as well as growth factor. These cells are renovated every 2 to 3 days in a process of proliferation and apoptosis. Disruptions in this process can lead to IBD by organisms breaking through the layer and triggering an immune response \citep{sartor_mechanisms_2006}.  Communication between the intestinal microbiota and the innate immune cells (such Macrophages and DCs) in the LP is done also by the epithelium via pattern recognition through toll like receptors (TLRs) and NOD-like receptors (NLRs). Regular lumen surveillance through these receptors allow physiological sampling of the host's microbiota content. This cooperation maintains host-microbiota mutualism \citep{slack_innate_2009} and intestinal homeostasis. Mutations in these receptors \citep{saruta_high-frequency_2009, natividad_commensal_2012} can lead to IBD by a continuous antigenic presentation that, in turn, would arise mucosal inflammation and deregulated proinflammatory pathways.

The next effectors in the innate response chain are the innate immune cells present in the LP. Among these we can find Antigen Presenting Cells (APCs) such as Macrophages and DC as well as the recently described ILCs \citep{spits_expanding_2011}. DCs sense antigens in the lumen and migrate from the LP to the mesenteric lymph nodes (MLNs) where they present the antigen to T cells, activating a tolerogenic response. It's been shown that a defective signaling between IECs and DCs can trigger a polarized Th1 response leading to an exacerbated inflammatory response \citep{rimoldi_intestinal_2005}. Other studies showing an increase of ILCs upon intestinal inflammation \citep{eken_il-23r_2014} also demonstrate the role of these cells in the disease. 

\subsubsection{Adaptive immune system}
\lipsum[1-1]


\subsection{Single cell technology}
\lipsum[1-1]

\end{document}
